\documentclass[USenglish,final,authoryear,12pt]{article}



\usepackage{amsmath}
\usepackage[a4paper]{geometry}
 \geometry{
 	a4paper,
 	total={210mm,297mm},
 	left=20mm,
 	right=20mm,
 	top=20mm,
 	bottom=20mm,
 }

\usepackage{sectsty}
\usepackage{cancel} %need to use this
\usepackage{babel}
%\allsectionsfont{\normalfont\sffamily\bfseries}
\usepackage{graphicx}
\usepackage{caption}
\usepackage{subcaption}
\usepackage{longtable}
%\begin{figure}[h]
%	\centering
%	\includegraphics[scale=0.5]{Image1.png}
%	\caption{Size of observable universe comparison between standard model of cosmology with inflation and without inflation [1].}
%\end{figure}

\begin{document}
\textbf{\LARGE Exercise Sheet 4}\newline
\section{Exercises}
\begin{enumerate}
	\item Using the Invoice table, find the number of invoices each customer has and filter for customers that don't have exactly 7 invoices.
	\item Using the InvoiceLine table, find the number of unique tracks purchased by each customer. Are there any customers who did not purchase exactly 38 tracks?
	\item Are there any customers who purchased a track more than once?
\end{enumerate}
\pagebreak

\section{Sample Solutions}
\begin{enumerate}
	\item SELECT "CustomerId",\newline
	COUNT(1) AS num\_invoices\newline
	FROM "Invoice"\newline
	GROUP BY "CustomerId"\newline
	HAVING COUNT(1) != 7;
	\item SELECT iv."CustomerId",\newline
	COUNT(DISTINCT "TrackId") AS unique\_tracks\_purchased\newline
	FROM "InvoiceLine" ivl\newline
	JOIN "Invoice" iv\newline
	ON ivl."InvoiceId" = iv."InvoiceId"\newline
	GROUP BY "CustomerId"\newline
	HAVING COUNT(DISTINCT "TrackId") != 38;
	\item SELECT iv."CustomerId",
	COUNT(DISTINCT "TrackId") AS unique\_tracks\_purchased,
	COUNT("TrackId") AS tracks\_purchased
	FROM "InvoiceLine" ivl
	JOIN "Invoice" iv
	ON ivl."InvoiceId" = iv."InvoiceId"
	GROUP BY "CustomerId"
	HAVING COUNT(DISTINCT "TrackId") != COUNT("TrackId");
\end{enumerate}
\end{document}

